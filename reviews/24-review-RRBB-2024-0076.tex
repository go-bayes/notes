% Options for packages loaded elsewhere
\PassOptionsToPackage{unicode}{hyperref}
\PassOptionsToPackage{hyphens}{url}
\PassOptionsToPackage{dvipsnames,svgnames,x11names}{xcolor}
%
\documentclass[
  singlecolumn]{article}

\usepackage{amsmath,amssymb}
\usepackage{iftex}
\ifPDFTeX
  \usepackage[T1]{fontenc}
  \usepackage[utf8]{inputenc}
  \usepackage{textcomp} % provide euro and other symbols
\else % if luatex or xetex
  \usepackage{unicode-math}
  \defaultfontfeatures{Scale=MatchLowercase}
  \defaultfontfeatures[\rmfamily]{Ligatures=TeX,Scale=1}
\fi
\usepackage[]{libertinus}
\ifPDFTeX\else  
    % xetex/luatex font selection
\fi
% Use upquote if available, for straight quotes in verbatim environments
\IfFileExists{upquote.sty}{\usepackage{upquote}}{}
\IfFileExists{microtype.sty}{% use microtype if available
  \usepackage[]{microtype}
  \UseMicrotypeSet[protrusion]{basicmath} % disable protrusion for tt fonts
}{}
\makeatletter
\@ifundefined{KOMAClassName}{% if non-KOMA class
  \IfFileExists{parskip.sty}{%
    \usepackage{parskip}
  }{% else
    \setlength{\parindent}{0pt}
    \setlength{\parskip}{6pt plus 2pt minus 1pt}}
}{% if KOMA class
  \KOMAoptions{parskip=half}}
\makeatother
\usepackage{xcolor}
\usepackage[top=30mm,left=20mm,heightrounded]{geometry}
\setlength{\emergencystretch}{3em} % prevent overfull lines
\setcounter{secnumdepth}{-\maxdimen} % remove section numbering
% Make \paragraph and \subparagraph free-standing
\ifx\paragraph\undefined\else
  \let\oldparagraph\paragraph
  \renewcommand{\paragraph}[1]{\oldparagraph{#1}\mbox{}}
\fi
\ifx\subparagraph\undefined\else
  \let\oldsubparagraph\subparagraph
  \renewcommand{\subparagraph}[1]{\oldsubparagraph{#1}\mbox{}}
\fi

\usepackage{color}
\usepackage{fancyvrb}
\newcommand{\VerbBar}{|}
\newcommand{\VERB}{\Verb[commandchars=\\\{\}]}
\DefineVerbatimEnvironment{Highlighting}{Verbatim}{commandchars=\\\{\}}
% Add ',fontsize=\small' for more characters per line
\usepackage{framed}
\definecolor{shadecolor}{RGB}{241,243,245}
\newenvironment{Shaded}{\begin{snugshade}}{\end{snugshade}}
\newcommand{\AlertTok}[1]{\textcolor[rgb]{0.68,0.00,0.00}{#1}}
\newcommand{\AnnotationTok}[1]{\textcolor[rgb]{0.37,0.37,0.37}{#1}}
\newcommand{\AttributeTok}[1]{\textcolor[rgb]{0.40,0.45,0.13}{#1}}
\newcommand{\BaseNTok}[1]{\textcolor[rgb]{0.68,0.00,0.00}{#1}}
\newcommand{\BuiltInTok}[1]{\textcolor[rgb]{0.00,0.23,0.31}{#1}}
\newcommand{\CharTok}[1]{\textcolor[rgb]{0.13,0.47,0.30}{#1}}
\newcommand{\CommentTok}[1]{\textcolor[rgb]{0.37,0.37,0.37}{#1}}
\newcommand{\CommentVarTok}[1]{\textcolor[rgb]{0.37,0.37,0.37}{\textit{#1}}}
\newcommand{\ConstantTok}[1]{\textcolor[rgb]{0.56,0.35,0.01}{#1}}
\newcommand{\ControlFlowTok}[1]{\textcolor[rgb]{0.00,0.23,0.31}{\textbf{#1}}}
\newcommand{\DataTypeTok}[1]{\textcolor[rgb]{0.68,0.00,0.00}{#1}}
\newcommand{\DecValTok}[1]{\textcolor[rgb]{0.68,0.00,0.00}{#1}}
\newcommand{\DocumentationTok}[1]{\textcolor[rgb]{0.37,0.37,0.37}{\textit{#1}}}
\newcommand{\ErrorTok}[1]{\textcolor[rgb]{0.68,0.00,0.00}{#1}}
\newcommand{\ExtensionTok}[1]{\textcolor[rgb]{0.00,0.23,0.31}{#1}}
\newcommand{\FloatTok}[1]{\textcolor[rgb]{0.68,0.00,0.00}{#1}}
\newcommand{\FunctionTok}[1]{\textcolor[rgb]{0.28,0.35,0.67}{#1}}
\newcommand{\ImportTok}[1]{\textcolor[rgb]{0.00,0.46,0.62}{#1}}
\newcommand{\InformationTok}[1]{\textcolor[rgb]{0.37,0.37,0.37}{#1}}
\newcommand{\KeywordTok}[1]{\textcolor[rgb]{0.00,0.23,0.31}{\textbf{#1}}}
\newcommand{\NormalTok}[1]{\textcolor[rgb]{0.00,0.23,0.31}{#1}}
\newcommand{\OperatorTok}[1]{\textcolor[rgb]{0.37,0.37,0.37}{#1}}
\newcommand{\OtherTok}[1]{\textcolor[rgb]{0.00,0.23,0.31}{#1}}
\newcommand{\PreprocessorTok}[1]{\textcolor[rgb]{0.68,0.00,0.00}{#1}}
\newcommand{\RegionMarkerTok}[1]{\textcolor[rgb]{0.00,0.23,0.31}{#1}}
\newcommand{\SpecialCharTok}[1]{\textcolor[rgb]{0.37,0.37,0.37}{#1}}
\newcommand{\SpecialStringTok}[1]{\textcolor[rgb]{0.13,0.47,0.30}{#1}}
\newcommand{\StringTok}[1]{\textcolor[rgb]{0.13,0.47,0.30}{#1}}
\newcommand{\VariableTok}[1]{\textcolor[rgb]{0.07,0.07,0.07}{#1}}
\newcommand{\VerbatimStringTok}[1]{\textcolor[rgb]{0.13,0.47,0.30}{#1}}
\newcommand{\WarningTok}[1]{\textcolor[rgb]{0.37,0.37,0.37}{\textit{#1}}}

\providecommand{\tightlist}{%
  \setlength{\itemsep}{0pt}\setlength{\parskip}{0pt}}\usepackage{longtable,booktabs,array}
\usepackage{calc} % for calculating minipage widths
% Correct order of tables after \paragraph or \subparagraph
\usepackage{etoolbox}
\makeatletter
\patchcmd\longtable{\par}{\if@noskipsec\mbox{}\fi\par}{}{}
\makeatother
% Allow footnotes in longtable head/foot
\IfFileExists{footnotehyper.sty}{\usepackage{footnotehyper}}{\usepackage{footnote}}
\makesavenoteenv{longtable}
\usepackage{graphicx}
\makeatletter
\def\maxwidth{\ifdim\Gin@nat@width>\linewidth\linewidth\else\Gin@nat@width\fi}
\def\maxheight{\ifdim\Gin@nat@height>\textheight\textheight\else\Gin@nat@height\fi}
\makeatother
% Scale images if necessary, so that they will not overflow the page
% margins by default, and it is still possible to overwrite the defaults
% using explicit options in \includegraphics[width, height, ...]{}
\setkeys{Gin}{width=\maxwidth,height=\maxheight,keepaspectratio}
% Set default figure placement to htbp
\makeatletter
\def\fps@figure{htbp}
\makeatother
% definitions for citeproc citations
\NewDocumentCommand\citeproctext{}{}
\NewDocumentCommand\citeproc{mm}{%
  \begingroup\def\citeproctext{#2}\cite{#1}\endgroup}
\makeatletter
 % allow citations to break across lines
 \let\@cite@ofmt\@firstofone
 % avoid brackets around text for \cite:
 \def\@biblabel#1{}
 \def\@cite#1#2{{#1\if@tempswa , #2\fi}}
\makeatother
\newlength{\cslhangindent}
\setlength{\cslhangindent}{1.5em}
\newlength{\csllabelwidth}
\setlength{\csllabelwidth}{3em}
\newenvironment{CSLReferences}[2] % #1 hanging-indent, #2 entry-spacing
 {\begin{list}{}{%
  \setlength{\itemindent}{0pt}
  \setlength{\leftmargin}{0pt}
  \setlength{\parsep}{0pt}
  % turn on hanging indent if param 1 is 1
  \ifodd #1
   \setlength{\leftmargin}{\cslhangindent}
   \setlength{\itemindent}{-1\cslhangindent}
  \fi
  % set entry spacing
  \setlength{\itemsep}{#2\baselineskip}}}
 {\end{list}}
\usepackage{calc}
\newcommand{\CSLBlock}[1]{\hfill\break\parbox[t]{\linewidth}{\strut\ignorespaces#1\strut}}
\newcommand{\CSLLeftMargin}[1]{\parbox[t]{\csllabelwidth}{\strut#1\strut}}
\newcommand{\CSLRightInline}[1]{\parbox[t]{\linewidth - \csllabelwidth}{\strut#1\strut}}
\newcommand{\CSLIndent}[1]{\hspace{\cslhangindent}#1}

\usepackage{booktabs}
\usepackage{longtable}
\usepackage{array}
\usepackage{multirow}
\usepackage{wrapfig}
\usepackage{float}
\usepackage{colortbl}
\usepackage{pdflscape}
\usepackage{tabu}
\usepackage{threeparttable}
\usepackage{threeparttablex}
\usepackage[normalem]{ulem}
\usepackage{makecell}
\usepackage{xcolor}
\input{/Users/joseph/GIT/latex/latex-for-quarto.tex}
\makeatletter
\@ifpackageloaded{caption}{}{\usepackage{caption}}
\AtBeginDocument{%
\ifdefined\contentsname
  \renewcommand*\contentsname{Table of contents}
\else
  \newcommand\contentsname{Table of contents}
\fi
\ifdefined\listfigurename
  \renewcommand*\listfigurename{List of Figures}
\else
  \newcommand\listfigurename{List of Figures}
\fi
\ifdefined\listtablename
  \renewcommand*\listtablename{List of Tables}
\else
  \newcommand\listtablename{List of Tables}
\fi
\ifdefined\figurename
  \renewcommand*\figurename{Figure}
\else
  \newcommand\figurename{Figure}
\fi
\ifdefined\tablename
  \renewcommand*\tablename{Table}
\else
  \newcommand\tablename{Table}
\fi
}
\@ifpackageloaded{float}{}{\usepackage{float}}
\floatstyle{ruled}
\@ifundefined{c@chapter}{\newfloat{codelisting}{h}{lop}}{\newfloat{codelisting}{h}{lop}[chapter]}
\floatname{codelisting}{Listing}
\newcommand*\listoflistings{\listof{codelisting}{List of Listings}}
\makeatother
\makeatletter
\makeatother
\makeatletter
\@ifpackageloaded{caption}{}{\usepackage{caption}}
\@ifpackageloaded{subcaption}{}{\usepackage{subcaption}}
\makeatother
\ifLuaTeX
  \usepackage{selnolig}  % disable illegal ligatures
\fi
\usepackage{bookmark}

\IfFileExists{xurl.sty}{\usepackage{xurl}}{} % add URL line breaks if available
\urlstyle{same} % disable monospaced font for URLs
\hypersetup{
  pdftitle={Review: RRBB-2024-0076 Quantifying Potential Selection Bias in Observational Research: Simulations and Analyses Exploring Religion and Depression Using a Prospective UK Cohort Study (ALSPAC)},
  pdfauthor={Joseph A. Bulbulia},
  colorlinks=true,
  linkcolor={blue},
  filecolor={Maroon},
  citecolor={Blue},
  urlcolor={Blue},
  pdfcreator={LaTeX via pandoc}}

\title{Review: RRBB-2024-0076 Quantifying Potential Selection Bias in
Observational Research: Simulations and Analyses Exploring Religion and
Depression Using a Prospective UK Cohort Study (ALSPAC)}
\author{Joseph A. Bulbulia}
\date{2024-04-02}

\begin{document}
\maketitle

\subsection{Decision}\label{decision}

Major revisions required before considering for publication.

\subsection{Summary}\label{summary}

This submission addresses the critical issue of sample and target
population mismatch, or ``selection bias.'' The authors specifically
focus on the implications of panel attrition in a longitudinal study of
pregnant women from Bristol in the early 1990s, examining the bias
introduced when estimating the effects of religiosity on depression. The
paper employs simulation and comparison of different datasets to
evaluate problems of mismatch.

Despite its merits, the manuscript contains methodological flaws that I
believe preclude publication in its current form. However, with
substantial revisions addressing the concerns outlined below, the study
has the potential for publication.

\subsection{Concerns and
Recommendations}\label{concerns-and-recommendations}

\begin{enumerate}
\def\labelenumi{\arabic{enumi}.}
\item
  \textbf{Methodological problems}: the study suffers from significant
  methodological issues. These issues must be thoroughly addressed in a
  revision.
\item
  \textbf{Misuse of regression coefficients}: The authors rely on
  regression coefficients to evaluate threats to external validity, but
  this is misguided. This approach fails to account for treatment effect
  modification, leading to inaccurate assessments of bias. Below, I
  demonstrate the problem using a simulation.
\item
  \textbf{Focus on marginal effect estimates}: a revised version of the
  paper should include a more robust analysis that accounts for the
  complexities of sample/target population mismatch and its implications
  estimating causal effects They should include interaction terms when
  simulating data, and they should specifically evaluate whether
  differences in the distribution of effect-modifiers in the sample and
  target population lead to differences between sample and target
  population causal effect estimates. Anything less is arguably
  unrealistic for longitudinal research in the study of religion.
\item
  \textbf{Encouragement}: despite the current shortcomings, the
  submission addresses an issue of considerable importance and
  widespread confusion. The authors should be encouraged to undertake
  the necessary revisions and resubmit their work. Appropriately
  considered findings will make a valuable contribution to the field.
\item
  \textbf{Acknowledgment of fallibility}: I offer my review with the
  hope that it will assist the authors in strengthening their paper. I
  apologise if I have overlooked important details in their work.
\end{enumerate}

\subsection{Simulation to clarify the
issues}\label{simulation-to-clarify-the-issues}

To understand why we must focus on marginal effect estimates, consider
the following simulation and analysis.

First, we load the \texttt{stdReg} library, which obtains marginal
effect estimates by simulating counterfactuals under different levels of
treatment (\citeproc{ref-sjuxf6lander2016}{Sjölander 2016}). If a
treatment is continuous, the levels can be specified.

We also load the \texttt{parameters} library, which creates nice tables
(\citeproc{ref-parameters2020}{Lüdecke \emph{et al.} 2020}).

\begin{Shaded}
\begin{Highlighting}[]
\CommentTok{\# to obtain marginal effects}
\FunctionTok{library}\NormalTok{(stdReg)}
\CommentTok{\# to create nice tables}
\FunctionTok{library}\NormalTok{(parameters)}
\end{Highlighting}
\end{Shaded}

Next, we write a function to simulate data for the sample and and target
populations.

We assume the treatment effect is the same in the sample and target
population. We will assume that the coefficient for the effect-modifier
and the coefficient for interaction are the same. We assume no
unmeasured confounding throughout the study. We assume only selective
attrition of one effect modifier such that the baseline population
differs from the sample population at the end of the study.

That is: \textbf{the distribution of effect modifiers is the only
respect in which the sample will differ from the target population.}

This function will generate data under a range of scenarios.\footnote{See
  documentation in the \texttt{margot} package: Bulbulia
  (\citeproc{ref-margot2024}{2024})}

\begin{Shaded}
\begin{Highlighting}[]
\CommentTok{\# function to generate data for the sample and population, }
\CommentTok{\# along with precise sample weights for the population, there are differences }
\CommentTok{\# in the distribution of the true effect modifier but no differences in the treatment effect }
\CommentTok{\# or the effect modification.all that differs between the sample and the population is }
\CommentTok{\# the distribution of effect{-}modifiers.}


\CommentTok{\# reproducability}
\FunctionTok{set.seed}\NormalTok{(}\DecValTok{123}\NormalTok{)}


\CommentTok{\# function based on margot::simulate\_ate\_data\_with\_weights, }
\CommentTok{\# see: https://go{-}bayes.github.io/margot/}

\NormalTok{simulate\_ate\_data\_with\_weights }\OtherTok{\textless{}{-}} \ControlFlowTok{function}\NormalTok{(}\AttributeTok{n\_sample =} \DecValTok{10000}\NormalTok{, }\CommentTok{\# sample n}
                                           \CommentTok{\#population n}
                                           \AttributeTok{n\_population =} \DecValTok{100000}\NormalTok{,}
                                           \CommentTok{\# prob of effect modifier in sample}
                                           \AttributeTok{p\_z\_sample =} \FloatTok{0.1}\NormalTok{,}
                                           \CommentTok{\#prob of effect modifier in the population}
                                           \AttributeTok{p\_z\_population =} \FloatTok{0.5}\NormalTok{,}
                                           \CommentTok{\# treatment effect}
                                           \AttributeTok{beta\_a =} \DecValTok{1}\NormalTok{,}
                                           \CommentTok{\# coef of intervention}
                                           \AttributeTok{beta\_z =} \FloatTok{2.5}\NormalTok{,}
                                           \CommentTok{\# coef  of effect{-}modifier}
                                           \AttributeTok{beta\_az =} \FloatTok{0.5}\NormalTok{,}
                                           \CommentTok{\# effect modification of a by z}
                                           \AttributeTok{noise\_sd =}\NormalTok{ .}\DecValTok{5}\NormalTok{) \{}

  \CommentTok{\# create sample data}
\NormalTok{  z\_sample }\OtherTok{\textless{}{-}} \FunctionTok{rbinom}\NormalTok{(n\_sample, }\DecValTok{1}\NormalTok{, p\_z\_sample) }\CommentTok{\# simulate data for sample z}
\NormalTok{  a\_sample }\OtherTok{\textless{}{-}} \FunctionTok{rbinom}\NormalTok{(n\_sample, }\DecValTok{1}\NormalTok{, }\FloatTok{0.5}\NormalTok{) }\CommentTok{\# for sample treatment}
  
  \CommentTok{\# simulate outcome}
\NormalTok{  y\_sample }\OtherTok{\textless{}{-}}\NormalTok{ beta\_a }\SpecialCharTok{*}\NormalTok{ a\_sample }\SpecialCharTok{+}\NormalTok{ beta\_z }\SpecialCharTok{*}\NormalTok{ z\_sample }\SpecialCharTok{+}\NormalTok{ beta\_az }\SpecialCharTok{*}\NormalTok{ (a\_sample }\SpecialCharTok{*}\NormalTok{ z\_sample) }\SpecialCharTok{+}
    \FunctionTok{rnorm}\NormalTok{(n\_sample, }\AttributeTok{mean =} \DecValTok{0}\NormalTok{, }\AttributeTok{sd =}\NormalTok{ noise\_sd)  }\CommentTok{\# use noise\_sd for the noise term}
  
  \CommentTok{\# put sample data in data frame}
\NormalTok{  sample\_data }\OtherTok{\textless{}{-}} \FunctionTok{data.frame}\NormalTok{(y\_sample, a\_sample, z\_sample)}
  
  \CommentTok{\# simulate population data, where the distribution of effect modifiers differs, but the treatment effect is the same}
\NormalTok{  z\_population }\OtherTok{\textless{}{-}} \FunctionTok{rbinom}\NormalTok{(n\_population, }\DecValTok{1}\NormalTok{, p\_z\_population)}
\NormalTok{  a\_population }\OtherTok{\textless{}{-}} \FunctionTok{rbinom}\NormalTok{(n\_population, }\DecValTok{1}\NormalTok{, }\FloatTok{0.5}\NormalTok{) }\CommentTok{\# same effect of a on y }
\NormalTok{  y\_population }\OtherTok{\textless{}{-}}\NormalTok{ beta\_a }\SpecialCharTok{*}\NormalTok{ a\_population }\SpecialCharTok{+}\NormalTok{ beta\_z }\SpecialCharTok{*}\NormalTok{ z\_population }\SpecialCharTok{+} 
\NormalTok{    beta\_az }\SpecialCharTok{*}\NormalTok{ (a\_population }\SpecialCharTok{*}\NormalTok{ z\_population) }\SpecialCharTok{+} \FunctionTok{rnorm}\NormalTok{(n\_population, }\AttributeTok{mean =} \DecValTok{0}\NormalTok{, }\AttributeTok{sd =}\NormalTok{ noise\_sd)  }\CommentTok{\# noise}
  
  \CommentTok{\# put population data in dataframe}
\NormalTok{  population\_data }\OtherTok{\textless{}{-}} \FunctionTok{data.frame}\NormalTok{(y\_population, a\_population, z\_population)}
  
  \CommentTok{\# simulate weighting based on z distribution difference}
\NormalTok{  weight\_z\_1 }\OtherTok{=}\NormalTok{ p\_z\_population }\SpecialCharTok{/}\NormalTok{ p\_z\_sample }\CommentTok{\# adjust weight for Z=1}
\NormalTok{  weight\_z\_0 }\OtherTok{=}\NormalTok{ (}\DecValTok{1} \SpecialCharTok{{-}}\NormalTok{ p\_z\_population) }\SpecialCharTok{/}\NormalTok{ (}\DecValTok{1} \SpecialCharTok{{-}}\NormalTok{ p\_z\_sample) }\CommentTok{\# adjust weight for Z=0}
\NormalTok{  weights }\OtherTok{\textless{}{-}} \FunctionTok{ifelse}\NormalTok{(z\_sample }\SpecialCharTok{==} \DecValTok{1}\NormalTok{, weight\_z\_1, weight\_z\_0)}
  
  \CommentTok{\# add weights to sample\_data}
\NormalTok{  sample\_data}\SpecialCharTok{$}\NormalTok{weights }\OtherTok{=}\NormalTok{ weights}
  
  \CommentTok{\# return list of data frames and weights}
  \FunctionTok{list}\NormalTok{(}\AttributeTok{sample\_data =}\NormalTok{ sample\_data, }\AttributeTok{population\_data =}\NormalTok{ population\_data)}
\NormalTok{\}}

\CommentTok{\# simulate the data {-}{-} you can use different parameters}
\NormalTok{data }\OtherTok{\textless{}{-}} \FunctionTok{simulate\_ate\_data\_with\_weights}\NormalTok{(}
  \AttributeTok{n\_sample =} \DecValTok{10000}\NormalTok{,}
  \AttributeTok{n\_population =} \DecValTok{100000}\NormalTok{,}
  \AttributeTok{p\_z\_sample =} \FloatTok{0.1}\NormalTok{,}
  \AttributeTok{p\_z\_population =} \FloatTok{0.5}\NormalTok{,}
  \AttributeTok{beta\_a =} \DecValTok{1}\NormalTok{,}
  \AttributeTok{beta\_z =} \FloatTok{2.5}\NormalTok{,}
  \AttributeTok{noise\_sd =} \FloatTok{0.5}
\NormalTok{)}
\end{Highlighting}
\end{Shaded}

Ok, we have generated both sample and population data.

Next, we verify that the distributions of effect modifiers differ in the
sample and in the target population:

\begin{Shaded}
\begin{Highlighting}[]
\CommentTok{\# obtain the generated data}
\NormalTok{sample\_data }\OtherTok{\textless{}{-}}\NormalTok{ data}\SpecialCharTok{$}\NormalTok{sample\_data}
\NormalTok{population\_data }\OtherTok{\textless{}{-}}\NormalTok{ data}\SpecialCharTok{$}\NormalTok{population\_data}


\CommentTok{\# check imbalance}
\FunctionTok{table}\NormalTok{(sample\_data}\SpecialCharTok{$}\NormalTok{z\_sample) }\CommentTok{\# type 1 is rare}
\end{Highlighting}
\end{Shaded}

\begin{verbatim}

   0    1 
9055  945 
\end{verbatim}

\begin{Shaded}
\begin{Highlighting}[]
\FunctionTok{table}\NormalTok{(population\_data}\SpecialCharTok{$}\NormalTok{z\_population) }\CommentTok{\# type 1 is common}
\end{Highlighting}
\end{Shaded}

\begin{verbatim}

    0     1 
49916 50084 
\end{verbatim}

Good, the distributions differ. The simulation is working as intended.

Next, consider the question: ``What are the differences in the
coefficients that we obtain from the study population at the end of
study, as compared with the target population?''

First, we obtain the coefficients for the sample. They are as follows:

\begin{Shaded}
\begin{Highlighting}[]
\CommentTok{\# model coefficients sample}
\NormalTok{model\_sample  }\OtherTok{\textless{}{-}}
  \FunctionTok{glm}\NormalTok{(y\_sample }\SpecialCharTok{\textasciitilde{}}\NormalTok{ a\_sample }\SpecialCharTok{*}\NormalTok{ z\_sample, }\AttributeTok{data =}\NormalTok{ sample\_data)}

\CommentTok{\# summary}
\NormalTok{parameters}\SpecialCharTok{::}\FunctionTok{model\_parameters}\NormalTok{(model\_sample, }\AttributeTok{ci\_method =} \StringTok{"wald"}\NormalTok{)}
\end{Highlighting}
\end{Shaded}

\begin{verbatim}
Parameter           | Coefficient |       SE |        95% CI | t(9996) |      p
-------------------------------------------------------------------------------
(Intercept)         |   -6.89e-03 | 7.38e-03 | [-0.02, 0.01] |   -0.93 | 0.350 
a sample            |        1.01 |     0.01 | [ 0.99, 1.03] |   95.84 | < .001
z sample            |        2.47 |     0.02 | [ 2.43, 2.52] |  104.09 | < .001
a sample × z sample |        0.51 |     0.03 | [ 0.44, 0.57] |   14.82 | < .001
\end{verbatim}

Ok, let's obtain the coefficients for the weighted regression of the
sample. Notice that the coefficients are virtually the same:

\begin{Shaded}
\begin{Highlighting}[]
\CommentTok{\# model the sample weighted to the population, again note that these coefficients are similar }
\NormalTok{model\_weighted\_sample }\OtherTok{\textless{}{-}}
  \FunctionTok{glm}\NormalTok{(y\_sample }\SpecialCharTok{\textasciitilde{}}\NormalTok{  a\_sample  }\SpecialCharTok{*}\NormalTok{ z\_sample,}
      \AttributeTok{data =}\NormalTok{ sample\_data,}
      \AttributeTok{weights =}\NormalTok{ weights)}

\CommentTok{\# summary}
\FunctionTok{summary}\NormalTok{(parameters}\SpecialCharTok{::}\FunctionTok{model\_parameters}\NormalTok{(model\_weighted\_sample, }\AttributeTok{ci\_method =}
                                       \StringTok{"wald"}\NormalTok{))}
\end{Highlighting}
\end{Shaded}

\begin{verbatim}
Parameter           | Coefficient |        95% CI |      p
----------------------------------------------------------
(Intercept)         |   -6.89e-03 | [-0.03, 0.01] | 0.480 
a sample            |        1.01 | [ 0.98, 1.04] | < .001
z sample            |        2.47 | [ 2.45, 2.50] | < .001
a sample × z sample |        0.51 | [ 0.47, 0.55] | < .001

Model: y_sample ~ a_sample * z_sample (10000 Observations)
Residual standard deviation: 0.494 (df = 9996)
\end{verbatim}

We might be tempted to infer that weighting wasn't relevant to the
analysis. However, we'll see that such an interpretation would be a
mistake.

Next, let us obtain model coefficients for the population. Note again
there is no difference -- only narrower errors owing to the large sample
size.

\begin{Shaded}
\begin{Highlighting}[]
\CommentTok{\# model coefficients population {-}{-} note that these coefficients are very similar. }
\NormalTok{model\_population }\OtherTok{\textless{}{-}}
  \FunctionTok{glm}\NormalTok{(y\_population }\SpecialCharTok{\textasciitilde{}}\NormalTok{ a\_population }\SpecialCharTok{*}\NormalTok{ z\_population, }\AttributeTok{data =}\NormalTok{ population\_data)}

\NormalTok{parameters}\SpecialCharTok{::}\FunctionTok{model\_parameters}\NormalTok{(model\_population, }\AttributeTok{ci\_method =} \StringTok{"wald"}\NormalTok{)}
\end{Highlighting}
\end{Shaded}

\begin{verbatim}
Parameter                   | Coefficient |       SE |        95% CI | t(99996) |      p
----------------------------------------------------------------------------------------
(Intercept)                 |    2.49e-03 | 3.18e-03 | [ 0.00, 0.01] |     0.78 | 0.434 
a population                |        1.00 | 4.49e-03 | [ 0.99, 1.01] |   222.35 | < .001
z population                |        2.50 | 4.49e-03 | [ 2.49, 2.51] |   556.80 | < .001
a population × z population |        0.50 | 6.35e-03 | [ 0.49, 0.51] |    78.80 | < .001
\end{verbatim}

Again, there is no difference. That is, we find that all model
coefficients are practically equivalent. The different distribution of
effect modifiers does not result in different coefficient values for the
treatment effect, the effect-modifier ``effect,'' or the interaction of
effect modifier and treatment.

Consider why this is the case: in a large sample where the causal
effects are invariant -- as we have simulated them to be -- we will have
good replication in the effect modifiers within the sample, so our
statistical model can recover the \emph{coefficients} for the population
-- no problem.

However, \textbf{in causal inference, we are interested in obtaining the
marginal effect of the treatment}. That is, we seek an estimate for the
counterfactual contrast in which everyone in a pre-specified population
was subject to one level of treatment compared with a counterfactual
condition in which everyone in a population was subject to another level
of the same treatment. \textbf{When the sample population differs in the
distribution of effect modifiers from the target population effect, the
marginal effect estimates will typically differ.}

To see this, we use the \texttt{stdReg} package to recover marginal
effect estimates, comparing (1) the sample ATE, (2) the true oracle ATE
for the population, and (3) the weighted sample ATE. We will use the
outputs of the same models above. The only difference is that we will
calculate marginal effects from these outputs. We will contrast a
difference from an intervention in which everyone receives treatment = 0
with one in which everyone receives treatment = 1, however, this choice
is arbitrary, and the general lessons apply irrespective of the
estimand.

First, consider this ATE for the sample population.

\begin{Shaded}
\begin{Highlighting}[]
\CommentTok{\# What inference do we draw?  We cannot say the models are unbiased for the marginal effect estimates. }
\CommentTok{\# regression standardisation }
\FunctionTok{library}\NormalTok{(stdReg) }\CommentTok{\# to obtain marginal effects }


\CommentTok{\# obtain sample ate}
\NormalTok{fit\_std\_sample }\OtherTok{\textless{}{-}}
\NormalTok{  stdReg}\SpecialCharTok{::}\FunctionTok{stdGlm}\NormalTok{(model\_sample, }\AttributeTok{data =}\NormalTok{ sample\_data, }\AttributeTok{X =} \StringTok{"a\_sample"}\NormalTok{)}

\CommentTok{\# summary}
\FunctionTok{summary}\NormalTok{(fit\_std\_sample,}
        \AttributeTok{contrast =} \StringTok{"difference"}\NormalTok{,}
        \AttributeTok{reference =} \DecValTok{0}\NormalTok{)}
\end{Highlighting}
\end{Shaded}

\begin{verbatim}

Formula: y_sample ~ a_sample * z_sample
Family: gaussian 
Link function: identity 
Exposure:  a_sample 
Reference level:  a_sample = 0 
Contrast:  difference 

  Estimate Std. Error lower 0.95 upper 0.95
0     0.00     0.0000       0.00       0.00
1     1.06     0.0101       1.04       1.08
\end{verbatim}

The treatment effect is given as a 1.06 unit change in the outcome
across the sample population, with a confidence interval from 1.04 to
1.08.

Next, we obtain the true (oracle) treatment effect for the population
under the same intervention.

\begin{Shaded}
\begin{Highlighting}[]
\DocumentationTok{\#\# note the population effect is different}

\CommentTok{\#obtain true ate}
\NormalTok{fit\_std\_population }\OtherTok{\textless{}{-}}
\NormalTok{  stdReg}\SpecialCharTok{::}\FunctionTok{stdGlm}\NormalTok{(model\_population, }\AttributeTok{data =}\NormalTok{ population\_data, }\AttributeTok{X =} \StringTok{"a\_population"}\NormalTok{)}

\CommentTok{\# summary}
\FunctionTok{summary}\NormalTok{(fit\_std\_population,}
        \AttributeTok{contrast =} \StringTok{"difference"}\NormalTok{,}
        \AttributeTok{reference =} \DecValTok{0}\NormalTok{)}
\end{Highlighting}
\end{Shaded}

\begin{verbatim}

Formula: y_population ~ a_population * z_population
Family: gaussian 
Link function: identity 
Exposure:  a_population 
Reference level:  a_population = 0 
Contrast:  difference 

  Estimate Std. Error lower 0.95 upper 0.95
0     0.00    0.00000       0.00       0.00
1     1.25    0.00327       1.24       1.26
\end{verbatim}

Behold, the true treatment effect is a 1.25 unit change in the
population, with a confidence bound between 1.24 and 1.26. This is well
outside the ATE that we obtain from the sample population!

Next, consider the ATE in the weighted regression, where the sample was
weighted to the target population's true distribution of effect
modifiers.

\begin{Shaded}
\begin{Highlighting}[]
\DocumentationTok{\#\# next try weights adjusted ate where we correctly assign population weights to the sample}
\NormalTok{fit\_std\_weighted\_sample\_weights }\OtherTok{\textless{}{-}}\NormalTok{ stdReg}\SpecialCharTok{::}\FunctionTok{stdGlm}\NormalTok{( model\_weighted\_sample, }
    \AttributeTok{data =}\NormalTok{ sample\_data, }
    \AttributeTok{X =} \StringTok{"a\_sample"}\NormalTok{)}

\CommentTok{\# this gives us the right answer}
\FunctionTok{summary}\NormalTok{(fit\_std\_weighted\_sample\_weights, }
    \AttributeTok{contrast =} \StringTok{"difference"}\NormalTok{, }
    \AttributeTok{reference =} \DecValTok{0}\NormalTok{)}
\end{Highlighting}
\end{Shaded}

\begin{verbatim}

Formula: y_sample ~ a_sample * z_sample
Family: gaussian 
Link function: identity 
Exposure:  a_sample 
Reference level:  a_sample = 0 
Contrast:  difference 

  Estimate Std. Error lower 0.95 upper 0.95
0     0.00     0.0000       0.00       0.00
1     1.25     0.0172       1.22       1.29
\end{verbatim}

\begin{Shaded}
\begin{Highlighting}[]
\CommentTok{\# Moral of the story. When we marginalise over the entire sample we need to weight estimates to the target population. }
\end{Highlighting}
\end{Shaded}

Good news, we find that we obtain the population-level causal effect
estimate with accurate coverage by weighting the sample to the target
population. So with appropriate weights, our results generalise from the
sample to the target population.

\subsection{Lessons}\label{lessons}

\begin{itemize}
\tightlist
\item
  Regression coefficients do not clarify the problem of sample/target
  population mismatch -- or selection bias as discussed in this
  manuscript.
\item
  The correct advice to investigators is that they should not rely on
  regression coefficients when evaluating the biases that arise from
  sample attrition. This advice applies to both methods that the authors
  use to investigate threats of bias. That is, to implement this advice,
  the authors must first take it.
\item
  Generally, observed data are insufficient for assessing threats.
  Observed data do not clarify structural sources of bias, nor do they
  clarify effect-modification in the full counterfactual data condition
  in which all receive the treatment and all do not receive the
  treatment (at the same level).
\item
  To properly assess bias, one would need access to the counterfactual
  outcome---what would have happened to the missing participants had
  they not been lost to follow-up or had they responded. Again, such
  counterfactual or ``full data'' are inherently unobservable
  (\citeproc{ref-vanderlaan2011}{Van Der Laan and Rose 2011}).
\item
  In simple settings like the one we just simulated, we may address the
  gap between the sample and target population using methods such as
  modelling the censoring (e.g., censoring weighting). However, we never
  know what setting we are in or whether it is simple---such modelling
  must be handled with care. There is a large and growing epidemiology
  literature on this topic (see, for example, Li \emph{et al.}
  (\citeproc{ref-li2023non}{2023})).
\item
  Matters become more complex when there is confounding and selection
  bias because the problem is not merely one of external validity but
  also internal validity (i.e.~obtaining valid causal effect estimates
  for the baseline sample). See for example Scharfstein \emph{et al.}
  (\citeproc{ref-scharfstein1999adjusting}{1999}); Laan and Robins
  (\citeproc{ref-laan2003unified}{2003}); Howe \emph{et al.}
  (\citeproc{ref-howe2016selection}{2016}) (note that Howe \emph{et al.}
  (\citeproc{ref-howe2016selection}{2016}), ``selection'' bias is
  defined as collider stratification bias, a variety of confounding bias
  and not simple sample and target population mismatch.)\\
\item
  Matters become more complex still in the presence of
  treatment-confounder feedback, for which only special methods are
  needed e.g. Rotnitzky \emph{et al.}
  (\citeproc{ref-rotnitzky2017multiply}{2017}) chapters in Laan and
  Gruber (\citeproc{ref-van2012targeted}{2012}).
\end{itemize}

\subsection{Suggestions for revision}\label{suggestions-for-revision}

\subsubsection{Assumptions and validity in specific
contexts}\label{assumptions-and-validity-in-specific-contexts}

\begin{itemize}
\tightlist
\item
  I think the authors must clarify the conditions under which their
  methodology is applicable. Theoretically, if they wanted to retain the
  current analysis, they might acknowledge that their approach could
  hold under unverifiable assumptions of Missing Completely At Random
  (MCAR) and the absence of interactions. Although such a commentary may
  be relevant for biological inquiries, such as vaccine efficacy, where
  these assumptions might be more plausible, it seems rather doubtful in
  the study of religions.
\end{itemize}

\subsubsection{Utility of regression
coefficients}\label{utility-of-regression-coefficients}

\begin{itemize}
\tightlist
\item
  I recommend critically examining how regression coefficients may
  mislead in assessing external validity. By illustrating the problem
  through simulations---along the lines suggested here---the authors
  could offer a valuable perspective. Such an approach would not only
  elucidate the methodological pitfalls but also enrich understanding of
  sample/target population mismatch. For example, the authors could
  re-examine the assumptions of previous studies.
\end{itemize}

\subsubsection{Address the ``Elephant in the
Room''}\label{address-the-elephant-in-the-room}

\begin{itemize}
\tightlist
\item
  The discussion should extend to the broader implications of
  sample/target population mismatch beyond the immediate context of
  pregnant women in Bristol during the early 1990s. Highlighting this
  issue encourages clarity about assessing the generalisability of
  causal effect estimates. To date, the small but growing literature in
  this area has not been careful on this point.
\end{itemize}

By addressing these points, a substantially revised manuscript has the
potential to make a substantive contribution to the literature on causal
inference and its application to the study of religion and culture more
generally.

\subsubsection{Small Points}\label{small-points}

\paragraph{1. Abstract accuracy}\label{abstract-accuracy}

\begin{itemize}
\tightlist
\item
  Non-random participation is not a unitary phenomenon, but a spectrum
  of challenges. These include defining or selecting the target
  population, varying participation rates, nested vs.~non-nested trial
  designs \& others. The authors might find the following works helpful:
  Dahabreh and Hernán (\citeproc{ref-dahabreh2019}{2019}) and Dahabreh
  \emph{et al.} (\citeproc{ref-dahabreh2021study}{2021}) for randomised
  trials and Bareinboim and Pearl
  (\citeproc{ref-bareinboim2013general}{2013}) for observational
  studies.
\end{itemize}

\paragraph{2. Abstract is unclear}\label{abstract-is-unclear}

\begin{itemize}
\tightlist
\item
  The abstract should promptly and clearly state the study's objective,
  focusing on the effect of church attendance on depression among the
  target population. The phrase ``despite non-random participation by
  the exposure and outcome'' is unclear because the meanings of these
  terms have yet to be defined.
\end{itemize}

\paragraph{3. Clarify the target population
throughout}\label{clarify-the-target-population-throughout}

\begin{itemize}
\tightlist
\item
  In abstract and throughout, clearly define the target population for
  ALSPAC at baseline, for example, as ``healthy pregnant women living in
  Bristol in the early 1990s''. Address the generalisation concern
  explicitly.
\end{itemize}

\paragraph{4. Revise redundant
statements}\label{revise-redundant-statements}

\begin{itemize}
\tightlist
\item
  For example, ``\ldots with attendance at a place of worship associated
  with continued participation\ldots{}'' This sounds like ``with
  attendance at a place of worship associated with attendance at a place
  of worship.''
\end{itemize}

\paragraph{5. Simplify terminology}\label{simplify-terminology}

\begin{itemize}
\tightlist
\item
  Consider consistently using a term such as ``composite religiosity''
  instead of the acronym \emph{RSBB} throughout the article for ease of
  understanding.
\end{itemize}

\paragraph{6. Clarify methodology}\label{clarify-methodology}

\begin{itemize}
\item
  The introduction to the simulation approach should explicitly address
  potential concerns about effect modification and how it influences the
  study's findings.
\item
  If you retain comparative data analysis (I think you should not),
  explain each step of the analysis clearly.
\end{itemize}

\paragraph{7. Clarify rationale for variable
selection}\label{clarify-rationale-for-variable-selection}

\begin{itemize}
\tightlist
\item
  The rationale for selecting a limited number of variables from ALSPAC
  should be clearly explained. Arguably, a richer dataset is needed to
  assess effect-modification. Moreover, confounding would need to be
  addressed as well---if internal validity fails, so too will external
  validity. The authors cannot lose sight of confounding bias if their
  interest is in external validity. Again, I suggest simulating data in
  which a mismatch arises even without confounding bias.
\end{itemize}

\paragraph{9. Directed Acyclic Graph -- not persuasive, and probably not
needed.}\label{directed-acyclic-graph-not-persuasive-and-probably-not-needed.}

\begin{itemize}
\tightlist
\item
  If you keep the DAG, make sure to avoid cycles (e.g., by indexing
  nodes by time); reconsider the credibility of the assumptions made,
  especially concerning effect modification by ethnicity (only through
  SEP and Marital Status)? Again, problems arise even with
  unconfoundedness, as demonstrated above, so addressing confounding
  bias is insufficient to address threats to generalisation from
  attrition. A simple contribution would set confounding bias to the
  side.
\end{itemize}

\subsection*{References}\label{references}
\addcontentsline{toc}{subsection}{References}

\phantomsection\label{refs}
\begin{CSLReferences}{1}{0}
\bibitem[\citeproctext]{ref-bareinboim2013general}
Bareinboim, E, and Pearl, J (2013) A general algorithm for deciding
transportability of experimental results. \emph{Journal of Causal
Inference}, \textbf{1}(1), 107--134.

\bibitem[\citeproctext]{ref-margot2024}
Bulbulia, JA (2024) \emph{Margot: MARGinal observational
treatment-effects}. Retrieved from
\url{https://go-bayes.github.io/margot/}

\bibitem[\citeproctext]{ref-dahabreh2021study}
Dahabreh, IJ, Haneuse, SJA, Robins, JM, \ldots{} Hernán, MA (2021) Study
designs for extending causal inferences from a randomized trial to a
target population. \emph{American Journal of Epidemiology},
\textbf{190}(8), 1632--1642.

\bibitem[\citeproctext]{ref-dahabreh2019}
Dahabreh, IJ, and Hernán, MA (2019) Extending inferences from a
randomized trial to a target population. \emph{European Journal of
Epidemiology}, \textbf{34}(8), 719--722.
doi:\href{https://doi.org/10.1007/s10654-019-00533-2}{10.1007/s10654-019-00533-2}.

\bibitem[\citeproctext]{ref-howe2016selection}
Howe, CJ, Cole, SR, Lau, B, Napravnik, S, and Eron Jr, JJ (2016)
Selection bias due to loss to follow up in cohort studies.
\emph{Epidemiology}, \textbf{27}(1), 91--97.

\bibitem[\citeproctext]{ref-van2012targeted}
Laan, MJ van der, and Gruber, S (2012) Targeted minimum loss based
estimation of causal effects of multiple time point interventions.
\emph{The International Journal of Biostatistics}, \textbf{8}(1).

\bibitem[\citeproctext]{ref-laan2003unified}
Laan, MJ, and Robins, JM (2003) \emph{Unified methods for censored
longitudinal data and causality}, Springer.

\bibitem[\citeproctext]{ref-li2023non}
Li, W, Miao, W, and Tchetgen Tchetgen, E (2023) Non-parametric inference
about mean functionals of non-ignorable non-response data without
identifying the joint distribution. \emph{Journal of the Royal
Statistical Society Series B: Statistical Methodology}, \textbf{85}(3),
913--935.

\bibitem[\citeproctext]{ref-parameters2020}
Lüdecke, D, Ben-Shachar, MS, Patil, I, and Makowski, D (2020)
Extracting, computing and exploring the parameters of statistical models
using {R}. \emph{Journal of Open Source Software}, \textbf{5}(53), 2445.
doi:\href{https://doi.org/10.21105/joss.02445}{10.21105/joss.02445}.

\bibitem[\citeproctext]{ref-rotnitzky2017multiply}
Rotnitzky, A, Robins, J, and Babino, L (2017) On the multiply robust
estimation of the mean of the g-functional. \emph{arXiv Preprint
arXiv:1705.08582}.

\bibitem[\citeproctext]{ref-scharfstein1999adjusting}
Scharfstein, DO, Rotnitzky, A, and Robins, JM (1999) Adjusting for
nonignorable drop-out using semiparametric nonresponse models.
\emph{Journal of the American Statistical Association},
\textbf{94}(448), 1096--1120.

\bibitem[\citeproctext]{ref-sjuxf6lander2016}
Sjölander, A (2016) Regression standardization with the R package
stdReg. \emph{European Journal of Epidemiology}, \textbf{31}(6),
563--574.
doi:\href{https://doi.org/10.1007/s10654-016-0157-3}{10.1007/s10654-016-0157-3}.

\bibitem[\citeproctext]{ref-vanderlaan2011}
Van Der Laan, MJ, and Rose, S (2011) \emph{Targeted Learning: Causal
Inference for Observational and Experimental Data}, New York, NY:
Springer. Retrieved from
\url{https://link.springer.com/10.1007/978-1-4419-9782-1}

\end{CSLReferences}



\end{document}
